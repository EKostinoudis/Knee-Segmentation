\documentclass[a4paper,12pt]{article}

% No intetation
\usepackage[parfill]{parskip}

\usepackage{amsmath}
\DeclareMathOperator*{\argmin}{\arg\!\min}
\usepackage{mathtools}
\DeclarePairedDelimiter\abs{\lvert}{\rvert}%
\DeclarePairedDelimiter\norm{\lVert}{\rVert}%

% Hyperlinks
\usepackage[hidelinks]{hyperref}

% Font
\usepackage{fontspec}
\setmainfont{GFS Artemisia}

% For bold
\usepackage{bm}

% English-Greek use
\usepackage{polyglossia}
% \setmainlanguage{english}
% \setotherlanguage{greek}
\setmainlanguage{greek}
\setotherlanguage{english}

\iffalse
\usepackage[greek, english]{babel}
\fi

% New page every section
\usepackage{titlesec}
\newcommand{\sectionbreak}{\clearpage}

% References
\usepackage[backend=biber,style=ieee]{biblatex}
\addbibresource{references.bib}
% \bibliography{references} 

\usepackage[justify]{ragged2e}
% \justifying


\tolerance=10000 
% \pretolerance=10000

\begin{document}

\tableofcontents

\section{Εισαγωγή}
\subsection{Διατύπωση του προβλήματος}
Η κατάτμηση είναι η διαδικασία διαμέρισης μίας εικόνας σε διάφορα ουσιαστικά
τμήματα. Σκοπός της κατάτμησης είναι η απλοποίηση ή/και η αλλαγή της
αναπαράστασης της εικόνας σε κάτι που είναι πιο σημασιολογικά σημαντικό και
είναι πιο εύκολο να αναλυθεί. Πιο συγκεκριμένα, η κατάτμηση εικόνας είναι η
διαδικασία ανάθεσης μίας ετικέτας σε κάθε εικονοστοιχείο της εικόνας έτσι ώστε
τα εικονοστοιχεία με την ίδια ετικέτα να έχουν ίδια χαρακτηριστικά.

Στην ιατρική απεικόνιση αυτά τα τμήματα αντιστοιχούν συχνά σε διαφορετικές
κατηγορίες ιστών, οργάνων, παθολογίες ή άλλες βιολογικά σχετιζόμενες δομές. Η
αυτοματοποιημένη κατάτμηση ιατρικών απεικονίσεων μπορεί να βοηθήσει τους
γιατρούς επιταχύνοντας τη διαδικασία διάγνωσης. Η καθημερινή δημιουργία πληθώρας
ιατρικών απεικονίσεων καθιστά τη χειροκίνητη κατάτμηση από ειδικούς όλο και
δυσκολότερη, λόγο του χρονικού διαστήματος που χρειάζεται η ανάλυση. Επομένως, η
ανάπτυξη αξιόπιστων, σταθερών και ακριβών τεχνικών για την αυτοματοποιημένη
κατάτμηση ιατρικών απεικονίσεων αποτελεί μία σημαντική πρόκληση.

\subsubsection{Κατάτμηση βάση ατλάντων}
Η κατάτμηση απεικονίσεων βάση ατλάντων αποτελεί την διαδικασία κατά την οποία
χρησιμοποιούνται απεικονίσεις που έχουν κατανεμηθεί από κάποιον ειδικό, ούτως
ώστε να επιτευχθεί η κατάτμηση της νέας απεικόνισης. Οι μέθοδοι αυτοί συνήθως
απαιτούν την χρήση καταχώρησης εικόνας (image registration) με σκοπό την
ευθυγράμμιση τού ή των ατλάντων στην εικόνα-απεικόνιση που πρόκειται να
κατανεμηθεί.

\subsubsection{Αντικείμενο διπλωματικής εργασίας}

Στην διπλωματική εργασία εφαρμόστηκαν μέθοδοι κατάτμησης ιατρικών απεικονίσεων
βάση ατλάντων με χρήση μηχανικής μάθησης. Οι απεικονίσεις αφορούν μαγνητικές
τομογραφίες σε γόνατα και οι περιοχές κατάτμησης αποτελούνται από τους αρθρικούς
χόνδρους και τα οστά. Τα δεδομένα που χρησιμοποιήθηκαν προέρχονται από το
Osteoarthritis Initiative Zuse Institute Berlin(OAI ZIB). Οι μέθοδοι που
χρησιμοποιήθηκαν προέρχονται από μεθόδους που έχουν χρησιμοποιηθεί για την
κατάτμηση περιοχών του εγκεφάλου.
TODO: references


\section{Θεωρητικό υπόβαθρο}
\iffalse
NOTES:
    !Graph-Based Framework for Label Fusion 
    !lasso
    registration
    dice, sse
    structural similarity measure (SSIM)
    φιλτρα
    leave-one-out cross validation
\fi

\subsection{Πλαίσιο συγχώνευσης κατηγοριών βασισμένο σε γράφους}
Το πλαίσιο συγχώνευσης κατηγοριών βασισμένο σε γράφους προτάθηκε από
το \cite{Zhang:1} ως μία μέθοδος μέσω της οποίας μπορούν να παραχθούν πολλές
υπάρχουσες μέθοδοι συγχώνευσης κατηγοριών.

Έστω το ζεύγος $\{(I_i,L_i),i=1,...,n\}$ όπου $I_i$ είναι η απεικόνιση ενός
άτλαντα, $L_i$ ο χάρτης των κατηγοριών της αντίστοιχης απεικόνισης και $n$ το
σύνολο των ατλάντων. Δοθείσας μία απεικόνισης $I$, δημιουργείται ο γράφος $G_i$
μεταξύ των εικονοστοιχείων $\bm{x}$ της δοθείσας εικόνας $I$ και του
εικονοστοιχείου $\bm{y}$ της απεικόνισης του άτλαντα $I_i$, μαζί με τα
αντίστοιχα βάρη $w_i(\bm{x},\bm{y})$, για $(\bm{x},\bm{y})\in\Omega^2$, όπου
$\Omega$ ο χώρος των απεικονίσεων. Με τη δημιουργία των βαρών του γράφου, η
συγχώνευση των κατηγοριών γίνεται για κάθε $\bm{x}$ της απεικόνισης εισόδου $I$
σύμφωνα με το τύπο:

\begin{equation}
    L(\bm{x})=\frac{ \sum_{i=1}^{n}  \sum_{\bm{y}\in\Omega}
                     w_i(\bm{x},\bm{y})L_i(\bm{y})}
    { \sum_{i=1}^{n}  \sum_{\bm{y}\in\Omega} w_i(\bm{x},\bm{y}) }
    , \forall \bm{x}\in\Omega
\end{equation}

$L$ είναι ο χάρτης των κατηγοριών της απεικόνισης εισόδου $I$.

% Lasso
\subsection{Ελάχιστα απόλυτος τελεστής συρρίκνωσης και επιλογή}
Ο ελάχιστα απόλυτος τελεστής συρρίκνωσης και επιλογής (least absolute shrinkage
and selection operator, Lasso) \cite{Lasso:1} είναι μία μέθοδος παλινδρόμησης.
Παλινδρόμηση είναι η μέθοδος πρόβλεψης της συμπεριφοράς μίας μεταβλητής
βασισμένη σε μία ή περισσότερες άλλες μεταβλητές. Έστω οι παρατηρήσεις
$(\bm{x_i},y_i), i=1,...,N$, όπου $\bm{x_i}=(x_{i1},...,x_{ip})$ οι $p$
μεταβλητές εισόδου και $y_i$ η μεταβλητή εισόδου για την $i$-οστή παρατήρηση. Αν
$\hat{\beta} = (\hat{{\beta}_1},...,\hat{{\beta}_p})^T$ και $\hat{\alpha}$ τα
αποτέλεσματα της παλινδρόμησης (συντελεστές παλινδρόμησης) τότε ο τελεστής
ορίζεται ως:

\begin{equation}\label{lassoEq1}
    (\hat{\alpha}, \hat{\beta}) = 
    \argmin{\left\{\sum_{i=1}^{N} {\left(\,y_i-\alpha- \sum_{j=1}^{p}
    \beta_{j}x_{ij}\right)}^2 \right\}} 
    \text{ subject to }
    \sum_{j=1}^{p} \abs{\beta_j} \leq t
\end{equation}

Το $t \geq 0$ είναι παράμετρος συντονισμού. Για κάθε τιμή του $t$ η λύση για το
$\alpha$ είναι $\hat{\alpha}=\overline{y}$ (όπου $\overline{y}$ είναι η μέση
τιμή του $y$). Επίσης μπορούμε να υποθέσουμε χωρίς την απώλεια της γενίκευσης
ότι $\hat{\alpha}=0$ ώστε να εξαλειφθεί το $\alpha$. Έστω $\bm{X} =
(\bm{x_1},...,\bm{x_N})^T$ και $y = (y_1,...,y_N)^T$ τότε η \eqref{lassoEq1}
μπορεί να γραφτεί ως:

\begin{equation*}\label{lassoEq2}
    \hat{\beta} = 
    \argmin{\left\{\frac{1}{N} \norm{\left(\,y - \bm{X} \beta\right)}_2^2
    \right\}} 
    \text{ subject to }
    \norm{\beta}_1 \leq t
\end{equation*}

Όπου $\norm{x}_p$ είναι η $p$-οστή νόρμα του $x$. Η εξίσωση μπορεί να γραφτεί
επίσης με την χρήση του πολλαπλασιαστή Lagrange ως:

\begin{equation}\label{lassoEq3}
    \hat{\beta} = 
    \argmin{\left\{\frac{1}{N} \norm{\left(\,y - \bm{X} \beta\right)}_2^2 +
    \lambda \norm{\beta}_1 \right\}} 
\end{equation}

Πολλές φορές οι στήλες του $\bm{X}$ κανονικοποιούνται, δηλαδή ισχύει:

\begin{equation*}
    \sum_{j=1}^{p} {\frac{x_{ij}}{N}} = 0
    \text{ , }
    \sum_{j=1}^{p} {\frac{x_{ij}^2}{N}} = 1
\end{equation*}


\iffalse
% TODO: Ίσως να το ϐαλω αυτό:
Το πλεονέκτημα του ελάχιστα απόλυτου τελεστή συρρίκνωσης και επιλογής είναι ότι
σε σχέση 
TODO: least squares, why lasso 
\fi

% registration
\subsection{Καταχώρηση εικόνας}
Καταχώρηση εικόνας (image registration) είναι η διαδικασία μετασχηματισμού
διαφορετικών δεδομένων σε ένα σύστημα συντεταγμένων. Η διαδικασία αυτή
επιδιώκει, μέσω του μετασχηματισμού αυτού, την επικάλυψη των κοινών
χαρακτηριστικών των δεδομένων. Τα δεδομένα μπορεί να είναι πολλαπλές
φωτογραφίες, δεδομένα από διαφορετικούς αισθητήρες, ώρες, βάθη και οπτικές
\cite{Registration:1}. Στις ιατρικές απεικονίσεις επιδιώκεται, μέσω του
μετασχηματισμού, αντίστοιχα εικονοστοιχεία των απεικονίσεων να αναπαριστούν
όμοια βιολογικά σημεία.

%TODO: εικόνα εδώ;
%TODO: κινούμενη και σταθερή εικόνα

Οι μέθοδοι της καταχώρησης εικόνας διαφέρουν ανάλογα με το δεδομένα που
πρόκειται να καταχωρηθούν \cite{Registration:2}. Για παράδειγμα υπάρχουν μέθοδοι
που βασίζονται στην ένταση των εικόνων και μέθοδοι που βασίζονται σε
χαρακτηριστικά τους.

\subsubsection{Μέση διαφορά τετραγώνων}

Στην οικογένεια των μεθόδων που βασίζονται στην ένταση των εικόνων, η μέτρηση
της ομοιότητας μεταξύ της κινούμενης και της σταθερής εικόνας αποτελεί ένα
δομικό στοιχείο της καταχώρησης εικόνας. Η μέτρηση αυτή αξιολογεί την
καταλληλότητα του μετασχηματισμού και χρησιμοποιείται για να υπολογιστεί η
παράγωγος της αξιολόγησης, ούτως ώστε να χρησιμοποιηθούν από τον αλγόριθμο
βελτιστοποίησης. Η επιλογή της μετρικής βασίζεται στο εκάστοτε πρόβλημα που
επιχειρεί να λύσει

Η μέση διαφορά τετραγώνων υπολογίζει τον μέσο όρο των τετραγώνων της διαφορά της
έντασης για κάθε εικονοστοιχείο μεταξύ της σταθερής και της κινούμενης εικόνας.
Η μετρική αυτή βασίζεται στην υπόθεση ότι η ένταση όμοιων σημείων είναι η ίδια
και στις δύο εικόνες και έχει την ίδια κατανομή και στις δύο εικόνες. Αυτό
σημαίνει ότι και οι εικόνες έχουν την ίδια μορφή. 

Έστω $X$ και $Y$ τρισδιάστατες εικόνες και $n_1, n_2, n_2$ ο αριθμός των
στοιχείων των εικόνων για κάθε διάσταση τότε η μέση διαφορά τετραγώνων είναι:


\begin{equation*}
    MSSD = \frac{1}{n_1 n_2 n_3} \sum_{i_1=1}^{n_1} \sum_{i_2=1}^{n_2} 
        \sum_{i_3=1}^{n_3} \left(\, {X\left(i_1,i_2,i_3\right)\, -
        Y\left(i_1,i_2,i_3\right)\,}
        \right)^2\, 
\end{equation*}


%TODO: μία εικόνα, μεταβλητές και περιγραφή τους

%TODO: Rigid registration

\newpage

\printbibliography


\end{document}
