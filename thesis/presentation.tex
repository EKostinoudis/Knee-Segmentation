\documentclass{beamer}

% Font
\usepackage{fontspec}
\setmainfont[Script=Greek]{GFS Artemisia}
\setsansfont[Script=Greek]{GFS Neohellenic}
\setmonofont[Script=Greek]{Noto Mono}

% English-Greek use
\usepackage{polyglossia}
% \setmainlanguage{english}
% \setotherlanguage{greek}
\setmainlanguage{greek}
\setotherlanguage{english}

% \newfontfamily\greekfont{GFS Artemisia}
% \let\greekfonttt\ttfamily


% Images
\usepackage{graphicx}
\graphicspath{{./images/}}
\usepackage[font={footnotesize,it}]{caption}
\usepackage[font={footnotesize}]{subcaption}
\renewcommand{\thesubfigure}{\Roman{subfigure}}
\usepackage{float}

\title[Short title] {Τίτλος διπλωματικής}

\institute 
{
    Αριστοτέλειο Πανεπιστήμιο Θεσσαλονίκης \\
    Τμήμα Ηλεκτρολόγων Μηχανικών και Μηχανικών Υπολογιστών \\
    Τομέας Ηλεκτρονικής και Υπολογιστών
}

\author {Κωστινούδης Ευάγγελος}

\logo{\includegraphics[height=2.0cm]{LogoAUTH72ppi.png}}
% \titlegraphic{\includegraphics[height=2cm]{LogoAUTH72ppi.png}}

\date{Απρίλιος 2021}

\makeatletter
\setbeamertemplate{title page} 
{
\vbox{}
% {\usebeamercolor[fg]{titlegraphic}\inserttitlegraphic\par}

{\hfill\usebeamercolor[fg]{titlegraphic}\inserttitlegraphic}
  \begin{centering}
    \begin{beamercolorbox}[sep=8pt,center]{institute}
      \usebeamerfont{institute}\insertinstitute
    \end{beamercolorbox}
    \begin{beamercolorbox}[sep=8pt,center]{title}
      \usebeamerfont{title}\inserttitle\par%
      \ifx\insertsubtitle\@empty%
      \else%
        \vskip0.25em%
        {\usebeamerfont{subtitle} \usebeamercolor[fg]{subtitle} \insertsubtitle\par}%
      \fi%

    \end{beamercolorbox}%
    \vskip1em\par
    \begin{beamercolorbox}[sep=8pt,center]{author}
      \usebeamerfont{author}\insertauthor
    \end{beamercolorbox}

    \small
    \begin{tabular}[t]{@{}l@{\hspace{2pt}}p{.32\textwidth}@{}}
        Επιβλέποντες: & \\
        Καθηγητής: & Θεοχάρης Ιωάννης \\
        Υποψήφιος Διδάκτορας: & Χαδουλός Χρήστος
    \end{tabular}

    \vfill
    \begin{beamercolorbox}[sep=7pt,center]{date}
      \usebeamerfont{date}\insertdate
    \end{beamercolorbox}%\vskip0.5em
  \end{centering}
}

\begin{document}

\begin{frame}

\maketitle

% \begin{tabular}[t]{@{}l@{\hspace{3pt}}p{.32\textwidth}@{}}
%     Επιβλέπων & & \\
%     & Καθηγητής: & Θεοχάρης Ιωάννης \\
%     & Υποψήφιος Διδάκτορας: & Χαδουλός Χρήστος
% \end{tabular}

\end{frame}

\begin{frame}
\frametitle{Table of Contents}
\tableofcontents
\end{frame}

\section{Εισαγωγή}

\begin{frame}
\frametitle{Ορισμοί (1/2)}
\begin{block}{Κατάτμηση}
Η κατάτμηση είναι η διαδικασία διαμέρισης μίας εικόνας σε διάφορα ουσιαστικά
τμήματα. Σκοπός της κατάτμησης είναι η απλοποίηση ή/και η αλλαγή της
αναπαράστασης της εικόνας σε κάτι που είναι πιο σημασιολογικά σημαντικό και
είναι πιο εύκολο να αναλυθεί.
\end{block} \pause

\begin{block}{Κατάτμηση βάση ατλάντων}
Η κατάτμηση απεικονίσεων βάση ατλάντων αποτελεί την διαδικασία κατά την οποία
χρησιμοποιούνται απεικονίσεις που έχουν κατανεμηθεί από κάποιον ειδικό, ούτως
ώστε να επιτευχθεί η κατάτμηση της νέας απεικόνισης.
\end{block} 
\end{frame}

\begin{frame}
\frametitle{Ορισμοί (2/2)}
\begin{block}{Αντικείμενο διπλωματικής}
Στην διπλωματική εργασία εφαρμόστηκαν μέθοδοι κατάτμησης ιατρικών απεικονίσεων
βάση ατλάντων με χρήση μηχανικής μάθησης. Οι απεικονίσεις αφορούν μαγνητικές
τομογραφίες σε γόνατα και οι περιοχές κατάτμησης αποτελούνται από τους
αρθρικούς χόνδρους και τα οστά. Τα δεδομένα που χρησιμοποιήθηκαν προέρχονται
από το Osteoarthritis Initiative Zuse Institute Berlin (OAI ZIB).
\end{block}
\end{frame}

% new section

\begin{frame}
\frametitle{Σύνοψη της διαδικασίας της κατάτμησης}
\begin{enumerate}
    \item<1-> Προεπεξεργασίας δεδομένων.Σκοπός της προεπεξεργασίας είναι να
        παραχθεί μία νέα απεικόνιση που θα έχει καλύτερα αποτελέσματα στην
        μάθηση από αυτά της αρχικής.
    \item<2-> Καταχώρηση (registration) ατλάντων. Η διαδικασία μετασχηματισμού
        διαφορετικών δεδομένων σε ένα σύστημα συντεταγμένων. Η διαδικασία αυτή
        επιδιώκει μέσω του μετασχηματισμού αυτού, την επικάλυψη των κοινών
        χαρακτηριστικών των δεδομένων.
    \item<3-> Επιλογή ατλάντων για την κατάτμηση.
    \item<4-> Κατάτμηση.
\end{enumerate}
\end{frame}

\section{Μεθοδολογία}

\begin{frame}
\frametitle{Προεπεξεργασία δεδομένων}
\begin{enumerate}
    \item<1-> Απαλοιφή θορύβου απεικόνισης μέσω ροής της καμπυλότητας

\end{enumerate}

\end{frame}

\end{document}
